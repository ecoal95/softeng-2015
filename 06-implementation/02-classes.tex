\subsection{Relación de las clases incluídas en cada componente}

Se ha dividido la aplicación en un total de seis componentes preparados para
el despliegue:

\begin{description}
  \item[Base] Este componente contiene los paquetes base de la infraestructura
    de la aplicación. Esto incluye las clases \texttt{View},
    \texttt{ViewParams}, \texttt{Request}, \texttt{Controller}, y resto de
    clases proporcionadas por el framework. Se considera un paquete estable, y
    por lo tanto la dependencia de él no conlleva ningún problema en el
    hipotético caso de que trataramos de descentralizar la aplicación.
  \item[Models] Este componente contiene todo lo relacionado con los modelos y
    el acceso a datos, que será compartido por el resto de componentes.
  \item[Admin] Este componente se encarga de la administración, y por seguridad
    se podría desplegar en otro puerto, u otro servidor menos accesible.
    Contendrá los controladores \texttt{VehiclesController},
    \texttt{CentersController}, \texttt{DotationsController} y
    \texttt{UsersControler}, y las vistas correspondientes.
  \item[Sessions] Este componente se encarga del control de sesiones, y contiene
    las clases \texttt{SessionsController} y sus vistas correspondientes.
  \item[Notifications] Este componente sólo contiene la clase
    \texttt{NotificationsSystem}, aunque usa las clases base y los modelos.
  \item[Emergency] Este componente contendrá la mayor parte de la aplicación,
    la parte \textit{pública}. Por lo tanto contiene todos los controladores y
    vistas restantes.
\end{description}
