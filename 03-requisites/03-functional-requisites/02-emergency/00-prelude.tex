\subsubsection{Paquete \texttt{Emergency}}

Este paquete recoge todos los casos de uso en los que actúa un usuario base de la aplicación (actor \texttt{User}.
\par

\paragraph{Consideraciones previas}
\begin{itemize}
    \item{Se asume (para evitar repetición innecesaria) que todos los requisitos funcionales del paquete excepto \texttt{[FRQ-03 Acceder al sistema]} tienen como \textbf{precondición: El usuario ha accedido correctamente al sistema.}}
    \item{Se considera que un usuario está \textbf{implicado en una emergencia} cuando:
        \begin{itemize}
            \item{Es un operario que ha registrado un aviso acerca de esa emergencia}
            \item{Es un personal de emergencia perteneciente a una dotación asignada a esa emergencia}
            \item{Es el responsable de un centro que tiene al menos una dotación asignada a esa emergencia}
        \end{itemize}
    }
\end{itemize}

\paragraph{Posibles requisitos a añadir}

Los siguientes requisitos/casos de uso no se han añadido formalmente por falta de tiempo y por que no los consideramos imprescindibles. No obstante se dejan aquí mencionados para una posible revaloración en futuras iteraciones:
\begin{description}
    \item[Salir del sistema] \hfill \\
        El usuario cierra la sesión
    \item[Ver últimas emergencias registradas] \hfill \\
        Este caso de uso iría asociado al actor \texttt{Phone Operator}, y facilitaría la tarea de comprobar si una emergencia ha sido previamente registrada. No se mostrarían todos los datos, sino sólo (por ejemplo) dirección y descripción, lo suficiente para identificar una emergencia rápidamente.
    \item[Comunicación entre dos usuarios cualquiera] \hfill \\
        Sería mediante un medio de comunicación que pudiera quedar registrado en el sistema, probablemente mensajes o chat. Por las características del problema la inmediatez debería de ser un requisito imprescindible de este sistema de comunicación.
\end{description}

